% Dokumentklassen sættes til memoir.
% Manual: http://ctan.org/tex-archive/macros/latex/contrib/memoir/memman.pdf
\documentclass[a4paper,oneside]{article}

% Ingen nummering af overskrifter
\setcounter{secnumdepth}{0}

% Danske udtryk (fx figur og tabel) samt dansk orddeling og fonte med
% danske tegn. Hvis LaTeX brokker sig over æ, ø og å skal du udskifte
% "utf8" med "latin1" eller "applemac".
\usepackage[utf8]{inputenc}
\usepackage[danish]{babel}
\usepackage[T1]{fontenc}

\usepackage{float}

% Matematisk udtryk, fede symboler, theoremer og fancy ting (fx kædebrøker)
\usepackage{amsmath,amssymb}
\usepackage{bm}
\usepackage{amsthm}

% Kodelisting. Husk at læse manualen hvis du vil lave fancy ting.
% Manual: http://mirror.ctan.org/macros/latex/contrib/listings/listings.pdf
\usepackage{pxfonts}
\usepackage{listings}
\lstset{language=VHDL,
	basicstyle=\ttfamily,
	keywordstyle=\bfseries,
	frame=single,
	numbers=left,
	breaklines=true,
	numberstyle=\footnotesize,
	tabsize=4, 
	showstringspaces=false
	%morekeywords={include, printf}
}

% Fancy ting med enheder og datatabeller. Læs manualen til pakken
% Manual: http://www.ctan.org/tex-archive/macros/latex/contrib/siunitx/siunitx.pdf
\usepackage{booktabs}

% Indsættelse af grafik.
\usepackage{graphicx,wrapfig}
%\usepackage[section,subsection,subsubsection]{extraplaceins}
\begin{document}

	
\newcommand{\HRule}{\rule{\linewidth}{0.1mm}} % Defines a new command for the horizontal lines, change thickness here
	
\begin{center}
	
\textsc{\LARGE Ingeniørhøjskolen Aarhus}\\[1.5cm] %

\textsc{\large Digital system design}\\[2.5cm] 
\HRule \\[0.8cm]
{\huge \bfseries \textsc{Journal 1}}\\[0.4cm]
\HRule \\[1.5cm]

\textsc{\large Hold 50}\\
\vspace{0.5 in}
\begin{tabular}{c c c}
	Cecilie Moriat &  Alexander Bennedsen & Lasse Stenhøj \\
	\textsl{201405949} & \textsl{201310498} & \textsl{201407500}
\end{tabular}

\vspace{2.5 in}

{\large\textit{\today}} \\[3cm]
\vfill % Fill the rest of the page with whitespace
\end{center} % Center everything on the page


\renewcommand{\lstlistingname}{Kode}

\tableofcontents

\newpage
\section{Øvelse 2}
\subsection{Opgave 1 - Half-adder}
\begin{enumerate}
	\item[1)]
	Architecture body i VHDL kan skrives på tre forskellige måder; dataflow, behavioral og structural. \\
	Half-adder beskrives i Dataflow style ved hjælp af direkte implementering af logiske gates. Skrivemåden gør det nemt at overføre programmet direkte til hardware og de logiske gates. \\
	I Behavioral style opskrives half-adderen ved brug af if/else statements. Dette giver en god forståelse af selve half-adderens funktion, men det er kompliceret at overføre det til logiske gates ud fra kodens syntax. \\
	I structural style laves et miks af de to ovenstående, da der først defineres en funktion for hver logisk gate som det ses i dataflow style, og derefter implementeres funktionerne i halfadder entity’en. Denne style er god at bruge, hvis man skal implementere en funktion flere gange.\\
	\item[2)]
	Et RTL view af de tre forskellige måder at skrive en half-adder på, giver følgende forskellige resultater:\\
\begin{figure}[h]
	\centering
\includegraphics[scale=0.8]{pictures/Oevelse1/Half_adder/Behavioral.JPG}
\caption{Half-adder - Behavioral RTL view}
\label{fig:HaBehavioralRTL}
\end{figure}

\begin{figure}[h]
	\centering
	\includegraphics[scale=0.8]{pictures/Oevelse1/Half_adder/dataflow.JPG}
	\caption{Half-adder - Dataflow RTL view}
	\label{fig:HaDataflowRTL}
\end{figure}

\begin{figure}[h]
	\centering
	\includegraphics[scale=0.8]{pictures/Oevelse1/Half_adder/Structural.JPG}
	\caption{Half-adder - Structural RTL view}
	\label{fig:HaStructuralRTL}
\end{figure}
	\newpage
	
	Ud fra ovenstående figurer ses det, at dataflow-style og behavioral-style "afkodes" på samme vis, hvor de blå bokse symboliserer en logisk funktion. I structural-style angiver de grønne bokse, også logiske funktioner, men disse er defineret af os, som hhv en AND-funktion og en XOR-funktion. Det er altså med structural-style at vi nemmest kan se, hvilke gates vi skal bruge i en fysisk opbygning af systemet.\\

	\item[3)]
	Følgende tre figurer viser en Timing simulation af hver style. \\
\begin{figure}[h]
	\centering
	\includegraphics[scale=0.6]{pictures/Oevelse1/Half_adder/Behavioral_timing_simulation.jpg}
	\caption{Half-adder - Behavioral Timing Simulation}
	\label{fig:HaBehavioralTimingSim}
\end{figure}
\begin{figure}[h]
	\centering
	\includegraphics[scale=0.6]{pictures/Oevelse1/Half_adder/Dataflow_timing_simulation.jpg}
	\caption{Half-adder - Dataflow Timing Simulation}
	\label{fig:HaDatalfowTimingSim}
\end{figure}
\begin{figure}[h]
	\centering
	\includegraphics[scale=0.6]{pictures/Oevelse1/Half_adder/Structural_timing_simulation.jpg}
	\caption{Half-adder - Structural Timing Simulation}
	\label{fig:HaStructuralTimingSim}
\end{figure}
	\newpage
	Som det ses på figurerne, forekommer der nogle spikes på funktionerne. Dette skyldes static hazard, da de "gates" vi bruger i vores half-adder, vil have en lille tidsforskydning fra hinanden, og dermed kan give forkerte resultater i brøkdelen af et nanosekund, som det eksempelvis ses når både a-signalet og b-signalet ændrer status.
	\item[4)] 
	For at undgå disse spikes laver vi en functional simulation. Denne slags simulering tager højde for static hazard, og optimerer diagrammet til at vise et "perfekt" resultat.\\
	Figur \ref{fig:HaDataflowFunctionalSim}, \ref{fig:HaBehavioralFunctionalSim} og \ref{fig:HaStructuralFunctionalSim} viser denne type simulering.\\
\begin{figure}[h]
	\includegraphics[scale=0.6]{pictures/Oevelse1/Half_adder/Dataflow_functional_simulation.jpg}
	\caption{Half-adder - Dataflow functional Simulation}
	\label{fig:HaDataflowFunctionalSim}
\end{figure}
\begin{figure}[h]
	\includegraphics[scale=0.6]{pictures/Oevelse1/Half_adder/Behavioral_functional_simulation.jpg}
	\caption{Half-adder - Behavioral functional Simulation}
	\label{fig:HaBehavioralFunctionalSim}
\end{figure}
\begin{figure}[h]
	\includegraphics[scale=0.6]{pictures/Oevelse1/Half_adder/Structural_functional_simulation.jpg}
	\caption{Half-adder - Structural functional Simulation}
	\label{fig:HaStructuralFunctionalSim}
\end{figure}
	\newpage
\end{enumerate}

\subsection{Opgave 2 - Full-adder}
	\flushleft
\begin{enumerate}
	\item[1)]
	Med to half-addere kan man lave en full-adder. Dette vil vi nu implementere i hhv. dataflow-style, behavioral-style og structural-style.\\
	\medskip
	\begin{lstlisting}[caption={Full-adder Dataflow VHDL kode},label={lst:FaDataflowCode}]
	library ieee;
	use ieee.std_logic_1164.all;
	
	entity full_adder_dataflow is
	port (a, b, carry_in : in std_logic;
	sum, carry_out : out std_logic);
	end full_adder_dataflow;
	
	architecture dataflow of full_adder_dataflow is
	
	signal s1, s2, s3 : std_logic;
	begin
	s1 <= a xor b; 
	sum <= s1 xor carry_in;
	s2 <= s1 and carry_in;
	s3 <= a and b;
	carry_out <= s2 or s3;
	
	end dataflow;
	\end{lstlisting}
	\medskip
	\begin{lstlisting}[caption={Full-adder Behavioral VHDL kode}, label={lst:FaBehavioralCode}]
	library ieee;
	use ieee.std_logic_1164.all;
	
	entity full_adder_behavioral is
	port (a, b, carry_in : in std_logic;
	sum, carry_out : out std_logic);
	end full_adder_behavioral;
	
	architecture behavioral of full_adder_behavioral is
	
	signal s1, s2, s3 : std_logic;
	begin
	fa: process (carry_in, a, b)
	begin
	if carry_in = '0'  then
	
	if a = '1' then
	sum <= not b;
	carry_out <= b;
	else
	sum <= b;
	carry_out <= '0';
	end if;
	else 
	sum <= a xnor b;
	carry_out <= a or b;
	end if;
	end process fa;
	
	end behavioral;
	\end{lstlisting}
	\medskip
	\begin{lstlisting}[caption={Full-adder Structural VHDL kode},label={lst:FaStructuralCode}]
	library ieee;
	use ieee.std_logic_1164.all;
	
	entity full_adder_structural is
	port (a, b, carry_in : in std_logic;
	sum, carry_out : out std_logic);
	end full_adder_structural;
	
	architecture structural of full_adder_structural is
	
	signal s1, s2, s3 : std_logic;
	begin
	
	ha1: entity work.half_adder_dataflow port map (a => a, b => b, sum => s1, carry_out => s3);
	ha2: entity work.half_adder_dataflow port map (a => s1, b => carry_in, sum => sum, carry_out => s2);
	or1: entity work.or_2 port map (i1 => s2, i2 => s3, o1 => carry_out);
	
	end structural;
	\end{lstlisting}
	\newpage
	\flushleft
		\item[2)]
		Med et RTL view kan vi se hvordan de tre forskellige koder vil blive omdannet til logiske gates.\\
		\begin{figure}[h]
		\centering
		\includegraphics[scale=0.7]{pictures/Oevelse1/Full_adder/Behavioral_RTL.JPG}
		\caption{Full-adder - Behavioral RTL view}
		\label{fig:FaBehavioralRTL}
	\end{figure}
	
	\begin{figure}[h]
		\centering
		\includegraphics[scale=0.7]{pictures/Oevelse1/Full_adder/dataflow_RTL.JPG}
		\caption{Full-adder - Dataflow RTL view}
		\label{fig:FaDataflowRTL}	
	\end{figure}
	
	\begin{figure}[h]
		\centering
		\includegraphics[scale=0.7]{pictures/Oevelse1/Full_adder/Structural_RTL.JPG}
		\caption{Full-adder - Structural RTL view}
		\label{fig:FaStructuralRTL}	
	\end{figure}
	
	\newpage
	\flushleft
		\item[3)]
		Til sidst laver vi en functional simulering for at se om vores tre full-adder koder opfører sig som vi ønsker.\\	

	\begin{figure}[h]
		\centering
		\includegraphics[scale=0.8]{pictures/Oevelse1/Full_adder/Dataflow_functional_simulation.jpg}
		\caption{Full-adder - Dataflow functional Simulation}
		\label{fig:FaDataflowFunctionalSim}
	\end{figure}
	\begin{figure}[h]
		\centering
		\includegraphics[scale=0.8]{pictures/Oevelse1/Full_adder/Behavioral_functional_simulation.jpg}
		\caption{Full-adder - Behavioral functional Simulation}
		\label{fig:FaBehavioralFunctionalSim}
	\end{figure}
	\begin{figure}[h]
		\centering
		\includegraphics[scale=0.6]{pictures/Oevelse1/Full_adder/Structural_functional_simulation.jpg}
		\caption{Full-adder - Structural functional Simulation}
		\label{fig:FaStructuralFunctionalSim}
	\end{figure}
	\newpage

\end{enumerate}


\newpage
\section{Øvelse 3}
\subsection{Opgave 1 - Four bit parallel adder}
\begin{enumerate}
	\item[1)]
Vi designer en 4 bit parallel adder i structural style ved hjælp af den dataflow-style four bit full adder, vi har lavet i øvelse 2 (se Kode \ref{lst:FaDataflowCode})


	\medskip
	\begin{lstlisting}[caption={Four bit parallel adder Structural VHDL kode},label={lst:4bitFaStructuralCode}]
	library ieee;
	use ieee.std_logic_1164.all;
	
	entity four_bit_full_adder is
	port (a: in std_logic_vector (3 downto 0);
	b: in std_logic_vector (3 downto 0);
	Cin: in std_logic;
	sum: out std_logic_vector (3 downto 0);
	Cout: out std_logic);
	
	end four_bit_full_adder;
	
	architecture structural of four_bit_full_adder is
	
	signal i1, i2, i3 : std_logic;
	begin
	full_ad1 : 	entity work.full_adder_dataflow port map (a => a(0), b => b(0), carry_in => cin, sum => sum(0), carry_out => i1 );
	full_ad2 : 	entity work.full_adder_dataflow port map (a => a(1), b => b(1), carry_in => i1, sum => sum(1), carry_out => i2 );
	full_ad3 : 	entity work.full_adder_dataflow port map (a => a(2), b => b(2), carry_in => i2, sum => sum(2), carry_out => i3 );
	full_ad4 : 	entity work.full_adder_dataflow port map (a => a(3), b => b(3), carry_in => i3, sum => sum(3), carry_out => Cout );
	
	end structural;
	\end{lstlisting}
	
\begin{figure}[H]
	\item[2)]
	Vi kan ved hjælp af RTL-viewer se, om vores full adders er forbundet korrekt:
	\centering
\includegraphics[scale=0.6]{pictures/Oevelse2/four_bit_full_adder_RTLview.jpeg}
\caption{Four bit parallel adder - Structural RTL view}
\label{fig:4bitFaBehavioralRTL}
\end{figure}

	\begin{figure}[H]
		\item[3)]
		Vi starter med at sætte bits til 00001111 samt cin 0 og får det forventede resultat:
		\centering
		\includegraphics[scale=0.5]{pictures/Oevelse3/00001111_cin0.jpg}
		\caption{Four bit parallel adder - 00001111, cin=0}
		\label{fig:4bitFa00001111cin0}
	\end{figure}

	\begin{figure}[H]
Vi sætter nu bits til 11110000 samt cin 1 og får det forventede resultat:
		\centering
		\includegraphics[scale=0.5]{pictures/Oevelse3/11110000_cin1.jpg}
		\caption{Four bit parallel adder - 11110000, cin=1}
		\label{fig:4bitFa11110000cin1}
	\end{figure}


	\begin{figure}[H]
			Vi sætter nu bits til 00010001 samt cin 1 og får det forventede resultat:
			\centering
			\includegraphics[scale=0.5]{pictures/Oevelse3/00010001_cin1.jpg}
			\caption{Four bit parallel adder - 00010001, cin=1}
			\label{fig:4bitFa00010001cin1}
		\end{figure}
		
\end{enumerate}
\newpage
\subsection{Opgave 2 - Four bit adder - using signed/unsigned logic}

\begin{enumerate}
	\item[1)]
	Vi laver en unsigned adder i dataflow style som vist på figur \ref{fig:4bitUnsignedAdder}. Da input og output skal være af std logic vector typen, og vi skal bruge + operatoren, bliver vi nødt til at konvertere til unsigned først. Se Kode \ref{lst:4bitUnsignedDataflowCode}\\
	
	\begin{figure}[H]
		\centering
		\includegraphics[scale=0.5]{pictures/Oevelse3/4bit_unsigned_adder.jpg}
		\caption{Four bit unsigned adder}
		\label{fig:4bitUnsignedAdder}
	\end{figure}
	
	\begin{lstlisting}[caption={Four bit unsigned adder Dataflow VHDL kode},label={lst:4bitUnsignedDataflowCode}]
	library ieee;
	use ieee.std_logic_1164.all;
	use ieee.numeric_std.all;
	
	entity unsigned_adder is
	port (a: in std_logic_vector (3 downto 0);
	b: in std_logic_vector (3 downto 0);
	sum: out std_logic_vector (3 downto 0));
	end unsigned_adder;
	
	architecture dataflow of unsigned_adder is
	begin
	
	sum <= std_logic_vector(unsigned(a) + unsigned(b));
	end dataflow;
	\end{lstlisting}
	
	\item[2)]
	Vi tester nu vores kode med en functional simulation:\\
	\begin{figure}[H]	
		\centering
		\includegraphics[scale=0.4]{pictures/Oevelse3/4bit_unsigned_adder_functional_simulation.jpeg}
		\caption{Four bit unsigned adder Functional simulation}
		\label{fig:4bitUnsignedAdderFuncSim}
	\end{figure}
\newpage
	\item[3)]
	Vi sætter nu bits til 1000 + 0100 og får det forventede resultat som det ses på Figur \ref{fig:4bitUnsignedAdder1100}\\
	\begin{figure}[H]
		
		\centering
		\includegraphics[width=0.8\textwidth]{pictures/Oevelse3/four_bit_unsigned_adder1.jpg}
		\caption{Four bit unsigned adder - 1000 + 0100}
		\label{fig:4bitUnsignedAdder1100}
	\end{figure}

	\item[4)]
	Vi ændrer nu koden så det bliver en signed adder som det ses i Kode \ref{lst:4bitsignedDataflowCode}. Vi tester det på vores DE2 board, og ser at der ingen forskel er på unsigned adderen og signed adderen som det ses på Figur \ref{fig:4bitSignedAdder1100}. Dette skyldes at selve bit'sne ikke er anderledes, men det er kun måden de skal tolkes på.
	
	\begin{lstlisting}[caption={Four bit signed adder Dataflow VHDL kode},label={lst:4bitsignedDataflowCode}]
	library ieee;
	use ieee.std_logic_1164.all;
	use ieee.numeric_std.all;
	
	entity unsigned_adder is
	port (a: in std_logic_vector (3 downto 0);
	b: in std_logic_vector (3 downto 0);
	sum: out std_logic_vector (3 downto 0));
	end unsigned_adder;
	
	architecture dataflow of unsigned_adder is
	begin
	
	sum <= std_logic_vector(unsigned(a) + unsigned(b));
	end dataflow;
	\end{lstlisting}
	
	\begin{figure}[H]
		
		\centering
		\includegraphics[width=0.8\textwidth]{pictures/Oevelse3/four_bit_signed_adder1.jpg}
		\caption{Four bit signed adder - 1000 + 0100}
		\label{fig:4bitSignedAdder1100}
	\end{figure}
\item[5)]
Vi laver nu vores unsigned adder om, så den også virker med et carry in, og leverer et carry out som vist på Figur \ref{fig:4bitUnsignedAdderCarry}. Vi benytter os af resize funktionen, samt laver nogle interne signaler, inden vi sender resultatet ud igen. Koden ses i Kode \ref{lst:4bitunsignedCarryDataflowCode}.
	\begin{figure}[H]
		\centering
		\includegraphics[scale=0.5]{pictures/Oevelse3/4bit_unsigned_adder_carry.jpg}
		\caption{Four bit unsigned adder with carry}
		\label{fig:4bitUnsignedAdderCarry}
	\end{figure}

	\begin{lstlisting}[caption={Four bit unsigned adder with carry Dataflow VHDL kode},label={lst:4bitunsignedCarryDataflowCode}]
	library ieee;
	use ieee.std_logic_1164.all;
	use ieee.numeric_std.all;
	
	entity unsigned_adder_carry is
	port (a: in std_logic_vector (3 downto 0);
	b: in std_logic_vector (3 downto 0);
	carry_in: in std_logic;
	carry_out : out std_logic_vector (0 downto 0);
	sum: out std_logic_vector (3 downto 0));
	end unsigned_adder_carry;
	
	architecture dataflow of unsigned_adder_carry is
	signal c : unsigned (3 downto 0);
	signal s : unsigned (4 downto 0);
	begin
	c <= "000" & carry_in;
	s <= resize(unsigned(a),5) + resize(unsigned(b),5) + resize(c,5) ;
	sum <= std_logic_vector(s(3 downto 0));
	carry_out <= std_logic_vector(s(4 downto 4));
	end dataflow;
	\end{lstlisting}

	\item[6)]
	Vi overfører vores adder til DE2 boardet. Her adderer vi 1110 + 0011 samt carry in = 1, og får det forventede resultat som ses på figur \ref{fig:4bitUnsignedAdderCarry10100}.
	\begin{figure}[H]
		\centering
		\includegraphics[width=0.8\textwidth]{pictures/Oevelse3/4bit_unsigned_adder_carry2.jpg}
		\caption{Four bit unsigned adder with carry}
		\label{fig:4bitUnsignedAdderCarry10100}
	\end{figure}
	
\end{enumerate}
\newpage
\medskip
\subsection{Opgave 3 - Concatenation}
\begin{enumerate}
	\item[1)]
		\begin{lstlisting}[caption={Concatenation kode},label={lst:ConcatenationCode}]
	library ieee;
	use ieee.Std_logic_1164.all;
	
	entity shift_div is
	port (a : in std_logic_vector(7 downto 0);
	a_shl,a_shr,a_ror: out std_logic_vector(7 downto 0));
	end shift_div; 
	
	architecture dataflow of shift_div is
	
	begin 
	a_shl <= a(6 downto 0) & '0';
	
	a_shr <= "00" & a(7 downto 2);
	
	a_ror <= a(2 downto 0) & a(7 downto 3);
	
	end dataflow ;

		\end{lstlisting}
	\item[2)]
Vi kan se på figur \ref{fig:concatenationRTL} fra RTL-viewer, at der ikke er nogen logiske elementer i kodestykket:
	\begin{figure}[H]
		\centering
		\includegraphics[scale=0.5]{pictures/Oevelse3/Concatenation_RTL.png}
		\caption{Concatenation RTL}
		\label{fig:concatenationRTL}
	\end{figure}

	
	\item[3)]
Vi overfører programmet til DE2-boardet. Inputtet a sættes som SW[7:0], output a-shl sættes til LEDR[7:0], a-shr sættes til LEDR[17:10] og a-ror sættes til LEDG[7:0]. Dette kan ses på figur \ref{fig:concatenation_DE2board}.

	\begin{figure}[H]
		\centering
		\includegraphics[scale=0.23]{pictures/Oevelse3/Concatenation_DE2board.jpg}
		\caption{Concatenation på DE2-board}
		\label{fig:concatenation_DE2board}
	\end{figure}

\end{enumerate}

\subsection{Opgave 4 - Subtype}
\begin{enumerate}
	\item[1)]
	Vi opskriver koden for en 4-bit subtractor som ses i kode \ref{lst:SubtractorCode}
	\begin{lstlisting}[caption={Subtractor kode},label={lst:SubtractorCode}]
	library ieee;
	use ieee.std_logic_1164.all;
	
	entity Subtypes is
	port (a,b : in std_logic;
	c  : out std_logic);
	end Subtypes;
	
	architecture dataflow of Subtypes is
	subtype bool is std_logic range '1' to 'Z';
	signal tmp : bool;
	begin 
	tmp<= 'U';
	c<= b and tmp;
	end dataflow;
		\end{lstlisting}
	\item[2)]
	
	Grunden til at vi får fejlen at U er udenfor rækkevidde ses på figur \ref{fig:stdulogicvalues} nedenfor:
	
		\begin{figure}[H]
			\centering
			\includegraphics[scale=0.23]{pictures/Oevelse3/Oevelse4_01Z.jpg}
			\caption{Stater og styrker for std\_ulogic værdier}
			\label{fig:stdulogicvalues}
		\end{figure}
	
	
	Derfor retter vi til koden
	
	\begin{lstlisting}[caption={Rettet subtractor kode},label={lst:SubtractorCode2}]
library ieee;
use ieee.std_logic_1164.all;

entity Subtypes is
port (a,b : in std_logic;
c  : out std_logic);
end Subtypes;

architecture dataflow of Subtypes is
subtype bool is std_logic range 'U' to 'Z';
signal tmp : bool;
begin 
tmp<= 'U';
c<= b and tmp;
end dataflow;
	\end{lstlisting}	

	
\end{enumerate}



\newpage
\section{Øvelse 4}
\subsection{Opgave 1: Combinational VHDL}
\begin{enumerate}
	\item[1)]
WHAT DA FUCK IS WRONG WITH UUU!
	
\end{enumerate}


\subsection{Opgave 2: Binary to Decimal Conversion}
\begin{enumerate}
	\item[1)]
	Vi har læst koden for BCD decoderen og er indforstået med hvad der fungerer hvordan, og hvorfor.
	
	\item[2)]
	
	
\end{enumerate}

\end{document}
